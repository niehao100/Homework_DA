\documentclass{ctexart}
%\usepackage{ctex}
%
%页眉页脚
\usepackage{fancyhdr}

\fancyhead[R]{\thepage}% 这是奇数页右页眉、偶数页左页眉
\fancyhead[L]{}
\chead{数字逻辑与处理器基础大作业}%这是中间页眉
\rhead{聂浩}
\cfoot{\thepage}
\pagestyle{fancy}
\usepackage{graphicx}
\usepackage{xcolor}
\usepackage{graphicx}
\usepackage{subfigure}
\usepackage{ amsmath}
\usepackage{listings} 
\usepackage{multirow}
\usepackage{enumerate}
\lstset{numbers=left, %设置行号位置
    numberstyle=\tiny, %设置行号大小
    keywordstyle=\color{blue}, %设置关键字颜色
    commentstyle=\color[cmyk]{1,0,1,0}, %设置注释颜色
    frame=single, %设置边框格式
    escapeinside=``, %逃逸字符(1左面的键),用于显示中文
    breaklines, %自动折行
    extendedchars=false, %解决代码跨页时,章节标题,页眉等汉字不显示的问题
    xleftmargin=2em,xrightmargin=2em, aboveskip=1em, %设置边距
    tabsize=4, %设置tab空格数
    showspaces=false %不显示空格
}

\title{数字逻辑与处理器基础大作业}
\author{聂浩~~ 无31~~ 2013011280}
\begin{document}
\maketitle
\section{处理器结构}
\subsection{试回答以下问题:}
\begin {enumerate}[a)]
             \item{由 RegDst 信号控制的多路选择器,输入 2 对应常数 31。 这里的 31 代表什么?在执行哪些指令时需要 RegDst 信号为 2?为什么?}

                 答:31代表\$ra 寄存器,表示在\$ra 里写下返回地址。jal指令、和jalr指令,因为在执行这两条命令时需要把返回地址写入\$ra。

             \item{由 ALUSrc1 信号控制的多路选择器,输入 1 对应的指令[10-6]是什么?在执行哪些指令时需要 ALUSrc1 信号为 1?为什么?}

                 答:偏移量。对应的指令是sll,srl,sra,因为它们需要确定移的位数。
             \item{由 MemtoReg 信号控制的多路选择器,输入 2 对应的是什么?在执行哪些指令时需要MemtoReg 信号为 2?为什么?}

                 答: 将下一条程序计数器+4之后的值写入内存。jal和jalr需要用到,因为它们需要把返回地址写入寄存器。

             \item{图中的处理器结构并没有 Jump 控制信号,取而代之的是 PCSrc 信号。 PCSrc 信号控制的多路选择器,输入 2 对应的是什么?在执行哪些指令时需要 PCSrc 信号为 2?为什么?}

                 答: 将寄存器读出的值写入程序计数器。jr和halr需要,因为它们需要从寄存器中读出跳转的地址。

                 \begin{table}[!hbp]
                     \centering
                     \begin{tabular}{|c|c|c|c|c|c|c|c|c|c|c|c|}
                         \hline
                         &\rotatebox{90}{PCSrc[1:0]}&\rotatebox{90}{Branch}&\rotatebox{90}{RegWrite}&\rotatebox{90}{RegDst[1:0]}&\rotatebox{90}{MemRead}&\rotatebox{90}{MemWrite}&\rotatebox{90}{MemtoReg[1:0]}&\rotatebox{90}{ALUSrc1}&\rotatebox{90}{ALUSrc2}&\rotatebox{90}{ExtOp}&\rotatebox{90}{LuOp}\\
                         \hline
                         lw      &0&0&1&0&1&0&1&0&1&1&0\\
                         \hline
                         sw      &0&0&0&x&0&1&x&0&1&1&0\\
                         \hline
                         lui     &0&0&1&0&0&0&0&0&1&1&1\\
                         \hline
                         add     &0&0&1&1&0&0&0&0&0&x&x\\
                         \hline
                         addu    &0&0&1&1&0&0&0&0&0&x&x\\
                         \hline
                         sub     &0&0&1&1&0&0&0&0&0&x&x\\
                         \hline
                         subu    &0&0&1&1&0&0&0&0&0&x&x\\
                         \hline
                         addi    &0&0&1&0&0&0&0&0&1&1&0\\
                         \hline
                         addiu   &0&0&1&0&0&0&0&0&1&1&0\\
                         \hline
                         and     &0&0&1&1&0&0&0&0&0&x&x\\
                         \hline
                         or      &0&0&1&1&0&0&0&0&0&x&x\\
                         \hline
                         xor     &0&0&1&1&0&0&0&0&0&x&x\\
                         \hline
                         nor     &0&0&1&1&0&0&0&0&0&x&x\\
                         \hline
                         andi    &0&0&1&0&0&0&0&0&1&0&0\\
                         \hline
                         sll     &0&0&1&1&0&0&0&1&0&x&x\\
                         \hline
                         srl     &0&0&1&1&0&0&0&1&0&x&x\\
                         \hline
                         sra     &0&0&1&1&0&0&0&1&0&x&x\\
                         \hline
                         slt     &0&0&1&1&0&0&0&0&0&x&x\\
                         \hline
                         sltu    &0&0&1&1&0&0&0&0&0&x&x\\
                         \hline
                         slti    &0&0&1&0&0&0&0&0&1&1&0\\
                         \hline
                         sltiu   &0&0&1&0&0&0&0&0&1&1&0\\
                         \hline
                         beq     &0&1&0&x&0&0&x&0&0&1&0\\
                         \hline
                         j       &1&0&0&x&0&0&x&x&x&x&x\\
                         \hline
                         jal     &1&0&1&2&0&0&2&x&x&x&x\\
                         \hline
                         jr      &2&0&0&x&0&0&x&x&x&x&x\\
                         \hline
                         halr    &2&0&1&1&0&0&2&x&x&x&x\\
                         \hline
                     \end{tabular}
                     \caption{控制器真值表}
                     \label{ture_t}
                 \end{table}
             \item{为什么需要 ExtOp 控制信号?什么情况下 ExtOp 信号为 1?什么情况下 ExtOp 信号为 0?}

                 答:判断扩展时是否做补码处理。当为andi等与立即数位操作时为0,在其他有offset和imm的情况下为1。
             \item {若想再多实现一条指令 nop(空指令),指令格式为全 0,需要如何修改处理器结构?}

                 答:不用修改,会直接执行sll指令。

         \end{enumerate}
         \subsection{根据对各控制信号功能的理解,填写真值表\ref{ture_t}}

         \section{完成控制器}
         \subsection{ CPU.v 实现了处理器的整体结构。 阅读 CPU.v,理解其实现方式。}
         \subsection{Control.v 是控制器模块的代码。 完成 Control.v。}

         补充的代码为
         \begin{lstlisting}[language=verilog]
         // Your code below
         assign 	PCSrc[1:0]=
         (OpCode == 6'h02 || OpCode==6'h03)?2'b01:
         (OpCode==0&&(Funct==6'h08 || Funct==6'h09))?2'b10:0;

         assign Branch=
         (OpCode== 6'h04)?1:0;

         assign RegWrite=
         (OpCode== 6'h2b || OpCode== 6'h04 || OpCode==6'h02)?0:
         (Funct == 6'h08 && OpCode==0)?0:1;

         assign RegDst=
         (OpCode== 6'h03)?2'b10:
         (OpCode== 6'h00)?2'b01:
         2'b00;

         assign MemRead=
         (OpCode== 6'h23)?1:0;

         assign MemWrite=
         (OpCode== 6'h2b)?1:0;

         assign MemtoReg=
         (OpCode== 6'h03 || Funct==6'h09)?2'b10:
         (OpCode== 6'h23)?1:0;

         assign ALUSrc1=
         (OpCode==0 && (Funct==6'h02 || Funct==6'h03 || Funct==0))?1:0;
         assign ALUSrc2=
         (OpCode==0 || OpCode==6'h04)?0:1;

         assign ExtOp=
         (OpCode==6'h0c)?0:1;

         assign LuOp=
         (OpCode==6'h0f)?1:0;

         // Your code above
         \end{lstlisting}
         \subsection{ 阅读 InstructionMemory.   
             v,根据注释理解指令存储器中的程序。这段程序执行足够长时间后会发生什么?此时寄存器\$a0~\$a3,
             \$t0~\$t2,\$v0~\$v1 中的值应是多少?写出计算过程。注意理解有符号数、无符号数以及各
             种进制表示的数之间的关系。如果已知某一时刻在某寄存器中存放着数 0xffffcfc7,能否判断出
         它是有符号数还是无符号数?为什么?}
         答:足够长的时间后会进入Loop: j Loop 死循环。

         ~~~~\$a0=0x00003039,~~~\$a1=0xffffd431,

         ~~~~\$a2=0xD4310000,~~~\$a3=0xffffd431,

         ~~~~\$t0=0xD4313039,~~~\$t1=0xfD431303,

         ~~~~\$t2=0xffffcfc7,~~~\$v0=1,\$v1=1.

         无法判断,因为无符号数和有符号数都可以存储为这样。

         \subsection  { 使用 ModelSim 等仿真软件进行仿真。 仿真顶层模块为 test\_cpu,这是一个 testbench,用于向 CPU 提供复位和时钟信号。 观察仿真结果中各寄存器和控制信号的变化。 回答以下问题:}
         \begin{enumerate}[a)] 
             \item{ PC 如何变化?}

                 答:不断加4,到10000时跳变到11000,之后继续加4直到101100不再改变。

             \item{Branch 信号在何时为 1?它引起了 PC 怎样的变化?}

                 答:400\~{}500ns。PC从10000跳变为11000而非10100.

             \item{100\~{}200ns 期间,PC 是多少?对应的指令是哪条?此时\$a1 的值是多少?200\~{}300ns 期间\$a1 的值是多少?为什么会这样?下一条指令立即使用到了\$a1 的值,会出现错误吗?为什么?}

                 答:0x00000004。对应的是addiu \$a1, \$zero, -11215。\$a1是0。200\~{}300ns
                 \$a1=0000\_0000\_0000\_0000\_1101\_0100\_0011\_0001。因为右十六位是-11215的补码。虽然仿真正确,但是可能会出现错误,因为regwrite是上升沿触发\$a1的值是在下一个时钟上升沿才改变的,而读取也是以下一个时钟上升沿读取的,立即使用可能会有冲突。

             \item{运行时间足够长之后(如 1100ns 时)寄存器\$a0~\$a3,\$t0~\$t2,\$v0~\$v1 中的值是多少?与你的预期是否一致?}

                 答:和预计值一致。
         \end{enumerate}
         \section{执行汇编程序}
         \subsection {如果第一行的 3 是任意正整数 n,这段程序能实现什么功能?Loop,sum,L1 各有什么作用?为
         每一句代码添加注释。}
         答:计算$\sum_{i=1}^ni$。Loop使得程序最终死循环,不退出,sum是求和函数的循环部分(类似for语句),L1是求和中的计算部分和函数的返回部分。
         \begin{lstlisting}[language=verilog]
         addi $a0, $zero, 3      \\a0初始值为3
         jal sum                 \\跳转到sum部分并保存当前PC到ra
         Loop:
         beq $zero, $zero, Loop  \\一直循环
         sum:
         addi $sp, $sp, -8       \\内存中设置栈空间
         sw $ra, 4($sp)          \\将ra入栈
         sw $a0, 0($sp)          \\将a0入栈
         slti $t0, $a0, 1        \\判断是否有a<0
         beq $t0, $zero, L1      \\a>=0则跳转至L1
         xor $v0, $zero, $zero   \\将v0初始化为0
         addi $sp, $sp, 8        \\栈指针加8
         jr $ra                  \\函数返回
         L1:
         addi $a0, $a0, -1       \\a0=a0-1
         jal sum                 \\跳到sum并保存当前PC到ra
         lw $a0, 0($sp)          \\重新读入保存的a0
         lw $ra, 4($sp)          \\重新读入保存的ra
         addi $sp, $sp, 8        \\栈指针加8
         add $v0, $a0, $v0       \\$v0=$v0+$a0
         jr $ra                  \\函数返回
         \end{lstlisting}

         \subsection{ 将这段汇编程序翻译成机器码。对于 beq 和 jal 语句中的 Loop,sum,L1,你是怎么翻译的?立即数-1、 -8 被翻译成了什么}
         (用 16 进制或 2 进制表示)?
         \begin{lstlisting}[language=verilog]
         case (Address[9:2])
         //addi $a0, $zero, 3
         8'd0:    Instruction <= {6'h08, 5'd0 , 5'd4 , 16'h3};
         //jal sum
         8'd1:    Instruction <= {6'h03,26'd3};
         //Loop:
         //beq $zero, $zero, Loop
         8'd2:    Instruction <= {6'h04,5'd0,5'd0,16'hffff};
         //sum:
         //addi $sp, $sp, -8
         8'd3:    Instruction <= {6'h08,5'd29,5'd29,16'hfff8};
         //sw $ra, 4($sp)
         8'd4:    Instruction <= {6'h2b,5'd29,5'd31,16'h04};
         //sw $a0, 0($sp)
         8'd5:    Instruction <= {6'h2b,5'd29,5'd4,16'h00};
         //slti $t0, $a0, 1
         8'd6:    Instruction <= {6'h0a,5'd4,5'd8,16'h01};
         //beq $t0, $zero, L1
         8'd7:    Instruction <= {6'h04,5'd8,5'd0,16'h03};
         //xor $v0, $zero, $zero
         8'd8:    Instruction <= {6'h0,5'd0,5'd0,5'd2,5'd0,6'h26};
         //addi $sp, $sp, 8
         8'd9:    Instruction <= {6'h08,5'd29,5'd29,16'h08};
         //jr $ra
         8'd10:    Instruction <= {6'h0,5'd31,15'd0,6'h08};
         //L1:
         //addi $a0, $a0, -1
         8'd11:    Instruction <= {6'h08,5'd4,5'd4,16'hffff};
         //jal sum
         8'd12:    Instruction <= {6'h03,26'd3};
         //lw $a0, 0($sp)
         8'd13:    Instruction <= {6'h23,5'd29,5'd4,16'h0};
         //lw $ra, 4($sp)
         8'd14:    Instruction <= {6'h23,5'd29,5'd31,16'h04};
         //addi $sp, $sp, 8
         8'd15:    Instruction <= {6'h08,5'd29,5'd29,16'h08};
         //add $v0, $a0, $v0
         8'd16:    Instruction <= {6'h0,5'd4,5'd2,5'd2,5'd0,6'h20};
         //jr $ra
         8'd17:    Instruction <= {6'h0,5'd31,15'd0,6'h08};
         default: Instruction <= 32'h00000000;
         endcase 
         \end{lstlisting}

         答: 使PC的值变为Loop,sum,L1对应的指令地址即可跳转。-1为0xffff,-8为0xfff8(十六位)。

         \subsection{修改 InstructionMemory.v,使 CPU 运行上面这段程序。 注意 case 语句的输入是地址的[9-
         2]比特。 仿真观察各控制信号和寄存器的变化。}
         \begin{enumerate}[a)]

             \item{ 运行时间足够长之后(如 5000ns 时),寄存器\$a0,\$v0 的值是多少?和你预期的程序功
                 能是否一致?}

                 答:\$a0 为0x00000003,\$v0为0x00000006.与预期一致。

             \item{观察、 描述并解释 PC,\$a0,\$v0,\$sp,\$ra 如何变化。}

                 答: 按照周期记录值:
                 PC:一共17条指令,所以只记录仅记录6~2位,其余都是0.

                 00000,00001,初始化。

                 00011,00100,00101,00110,00111,01011,01100;

                 00011,00100,00101,00110,00111,01011,01100;

                 00011,00100,00101,00110,00111,01011,01100;三次递归,直到a0=0

                 00011,00100,00101,00110,00111,

                 01000,01001,01010;初始化v0,然后第一次返回。

                 01101,01110,01111,10000,10001;v0=a0+v0,然后返回,这样执行三次

                 01101,01110,01111,10000,10001;

                 01101,01110,01111,10000,10001,

                 00010;进入Loop死循环

                 \$a0依次为(10进制)3,2,1,0,1,2,3.a0先每次递归减一,之后出栈则依次恢复上一层的值。

                 \$v0依次为(10进制)0,1,3,6.最后出栈时,每出一次栈,v0=v0+a0.

                 \$sp依次为 0x00000000,0xfffffff8,0xfffffff0,0xffffff78,0xffffff70,之前都为入栈(四次)。

                 0xffffff78,0xfffffff0,0xfffffff8,0x00000000.出栈。

                 \$ra同样只记录6~2 位。

                 00000,00010,01101,00010.第一次um时ra变为00010,之后因为数次jal sum都是指令12,所以此时ra一直为01101,数次出栈ra也不会改变。最后一次出栈使得ra变为0010,之后在死循环中ra不再改变。
         \end{enumerate}
         \end{document}
