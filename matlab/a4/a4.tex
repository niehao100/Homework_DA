%#-*- coding: utf-8 -*-
\documentclass{ctexart}
%
%页眉页脚
\usepackage{geometry}
\geometry{left=2.5cm,right=2.5cm,top=2.5cm,bottom=2.5cm}
\usepackage{xcolor}
\usepackage{graphicx}
\usepackage{amsmath}
\usepackage{url}
\usepackage{enumerate}
\usepackage{subfigure}
\usepackage{listings}
\usepackage[colorlinks,linkcolor=black]{hyperref}%书签
\usepackage{fancyhdr}

\fancyhead[R]{\thepage}%这是奇数页右页眉、偶数页左页眉
\fancyhead[L]{}
\chead{MATLAB综合实验之图像处理}%这是中间页眉
\pagestyle{fancy}
\lstset{numbers=left,%设置行号位置
   numberstyle=\tiny, %设置行号大小
   keywordstyle=\color{blue}, %设置关键字颜色
   commentstyle=\color[cmyk]{1,0,1,0}, %设置注释颜色
   frame=single, %设置边框格式
   breaklines, %自动折行
   extendedchars=false, %解决代码跨页时,章节标题,页眉等汉字不显示的问题
   xleftmargin=1.5em,xrightmargin=1.5em, aboveskip=1em, %设置边距
   tabsize=4, %设置tab空格数
   showspaces=false %不显示空格
}
%中文
\usepackage{xeCJK}
%字体设置
\usepackage{indentfirst}
\setlength{\parindent}{2em}%首行缩进
\renewcommand\thesubsection{(\arabic{subsection})}
\renewcommand\thesubsubsection{(\alph{subsubsection})}

\title{MATLAB综合实验之连连看\footnote{所有的.m文件均采用utf8编码,windows版matlab中打开可能会出现中文乱码的情况,请用其它编辑器打开}}
\author{聂浩~~无31~~ 2013011280}
\date{\today}
\begin{document}
\maketitle
\section{制作自己的连连看}
\subsection{
MBATLAB 为 环境下,设置当前路径为  linkgame行 ,运行  linkgame (打开
linkgame.fig键 或右键 p linkgame.p  点“运行” ) ,熟悉游戏。 (上述程序己经过
的测试。)}
\subsection{
注意linkgame目录下有个detect.p。它的功能是检测块是否可以消除。
现在请你把它移动到其他文件夹或删掉!把然后把linkgame\\reference目录
下的detect.m复制到linkgame。目录下。detect.m文件中是tdetect函数,
函数以图像块的索号矩阵与要判断的两个块的下标为输入,如果两个块能消掉则
输出11,否则输出00。请根据文件中的注释提示,实现判断块是否可以消除的功
能。写完后再次运行linkgame,检验游戏是否仍然可以正确运行,当你的程序
的判断结果有误时,在游戏界面右下角会有提示。(注意:当detect.p文件存
在时,detect.m文件将不会被执行,所以测试时一定要移走detect.p)}
利用讲义中的十字判断法进行了判断,使用求和的方式检测通路上是否都是零。

值得说明的是边缘检测是比较麻烦;当两个块相等时才应该进行下一步判断,这里值得注意。具体可见check.m
代码如下:
detect.m
\lstinputlisting[language=matlab]{linkgame/detect.m}
check.m
\lstinputlisting[language=matlab]{linkgame/check.m}
\subsection{
你一定发现了“外挂”模式,是不是很有趣?逐一自动消除所有的块的功
由能是由link的目录的omg.p实现的。现在请你把它也删掉!然后把
link\\reference目录下的omg.m复制到link目录下。omg.m文件的注释中对
输入输出变量做了详细说明,请以这个文件为基础,实现逐一自动消除所有块的
功能。(同上题要移走omg.p文件。)写完后再次运行linkgame,检验自动模
式是否正确。(在自动点击过程中可接F12)
}
这里有两种思路,一种是从外圈逐渐向内消;另一种是逐一消除同一种块,多次循环即可,这也是手动消除的常用方法,这里选用这种。有趣的是,这并不能保证一定能够解出来所有有解的连连看,一些精巧设计的连连看需要按照一定的顺序消除,甚至是唯一的顺序才能解开——实际上在本代码的消除顺序下也无法解开第二部分图像中的连连看。至于是否能设计一种一定能找到合理解的算法,这涉及到了图论和组合数学的知识,不进行详细讨论。本部分代码如下(omg.m):
\lstinputlisting[language=matlab]{linkgame/omg.m}

\subsection{自由发挥}
这里自己设计了一个连连看矩阵生成程序,生成一个随机矩阵,然后用上一问的方法检测是否有解。这里没有进行难度判断的代码实现,但设想为使用不同的图像判断顺序对生成的矩阵多判断即便,得出解的判断顺序越少难度越高。
代码如下generate.m
\lstinputlisting[language=matlab]{linkgame/generate.m}
\section{攻克别人的连连看}
\subsection{ 
在MATLAB环境下,将路径设置至process文件夹下。对游戏区域的屏
幕截图(灰度图像)graygroundtruth进行分剖,提取出所有图像分块。在一个
figure中用subplot方式按照原始页序绘出所有的图像分块}
这里现
